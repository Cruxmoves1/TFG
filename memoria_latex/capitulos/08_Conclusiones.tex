%%%%%%%%%%%%%%%%%%%%%%%%%%%%%%%%%%%%%%%%%%%%%%%%%%%%%%%%%%%%%%%%%%%%%%%%
% TFG: Vigilancia Tecnológica y Minería de Opiniones en RRSS
% Escuela Técnica Superior de Ingenierías Informática y de Telecomunicación
% Realizado por: Miguel Keane Cañizares
% Contacto: miguekeca@correo.ugr.es 
%%%%%%%%%%%%%%%%%%%%%%%%%%%%%%%%%%%%%%%%%%%%%%%%%%%%%%%%%%%%%%%%%%%%%%%%

\chapter{Conclusiones}

Durante la realización de este trabajo, he llegado a la conclusión de lo útil y completo que es el lenguaje de programación Python, y como, gracias a los aportes de la comunidad, se han desarrollado librerías muy potentes que permiten realizar tareas muy complejas con cierta sencillez. Esto es posible gracias a la distribución de software libre que nos permite apoyarnos en trabajos ya realizados construyendo cada vez un desarrollo mayor.

Por ello pongo a disposición de quién lo desee este proyecto, bajo una licencia GNU General Public License, la cual es abierta para el que quiera usarlo. El proyecto se puede encontrar en el repositorio Github: 

\href{https://github.com/mikykeane/TFG/}{https://github.com/mikykeane/TFG/}


También me ha hecho valorar aún más la importancia de la Big Data. De cómo vivimos en un mundo cada vez más público, ddónde el tráfico de datos personales cobrea una gran relevancia, cedemos nuestra información para obtener servicios online, la cual será comprada por grandes empresas para analizarla y exprimirla todo lo posible. Esto abre un mundo de posibilidades, algunas excitantes y otras aterradoras, pues el progreso en sí no es ni bueno ni malo, solo el uso que hagamos del mismo puede estar sujeto a la moralidad. 